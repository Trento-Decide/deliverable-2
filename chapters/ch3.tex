% !TeX spellcheck = it_IT
\chapter{Front-end}

\section{Interfaccia di Autenticazione e Registrazione}

\begin{figure}[H]
    \centering
    \includegraphics[width=0.50\textwidth]{img/frontend/accedi.png}
    \caption{Schermata di accesso (Login). L'interfaccia evidenzia le opzioni di accesso tramite SPID e CIE, oltre al form per email e password. È presente anche un link diretto alla registrazione per i nuovi utenti.}
    \label{fig:accedi}
\end{figure}

\begin{figure}[H]
    \centering
    \includegraphics[width=0.65\textwidth]{img/frontend/registrati.png}
    \caption{Interfaccia di registrazione. La UI guida l'utente all'utilizzo dei sistemi di identità digitale (SPID/CIE) per completare l'iscrizione.}
    \label{fig:registrati}
\end{figure}

\section{Sezione Guida}

\begin{figure}[H]
    \centering
    \includegraphics[width=0.50\textwidth]{img/frontend/guida.png}
    \caption{Sezione 'Come funziona'. La guida spiega la distinzione tra utente Ospite e Registrato e mostra il ciclo di vita della proposta, il meccanismo di voto positivo/negativo, la partecipazione ai sondaggi e una sintesi delle regole di moderazione.}
    \label{fig:guida}
\end{figure}

\section{Interfaccia di Ricerca Avanzata}

\begin{figure}[H]
    \centering
    \includegraphics[width=0.85\textwidth]{img/frontend/ricerca-avanzata.png}
    \caption{Interfaccia di Ricerca Avanzata. Sono visibili i widget per il filtraggio parametrico (Categoria, Autore, Data) e il menu di ordinamento dei risultati.}
    \label{fig:ricerca-avanzata}
\end{figure}

\begin{figure}[H]
    \centering
    \includegraphics[width=0.9\textwidth]{img/frontend/cerca.png}
    \caption{Pagina Cerca con barra laterale di filtraggio, vengono mostrate le proposte di alice con categoria "Innovazione".}
    \label{fig:cerca}
\end{figure}

\section{Pagina del Regolamento}

\begin{figure}[H]
    \centering
    \includegraphics[width=0.60\textwidth]{img/frontend/regolamento.png}
    \caption{Pagina del Regolamento. La vista presenta i requisiti per i partecipanti e le regole di moderazione attive della piattaforma.}
    \label{fig:regolamento}
\end{figure}

\section{Homepage}

\begin{figure}[H]
    \centering
    \includegraphics[width=1.0\textwidth]{img/frontend/home.png}
    \caption{Homepage di un utente loggato. Sono visibili le sezioni per i sondaggi a scadenza e le proposte recenti, oltre ai contatori statistici della piattaforma.}
    \label{fig:home}
\end{figure}

\section{Creazione e Visualizzazione di Proposte e Sondaggi}

\begin{figure}[H]
    \centering
    \includegraphics[width=0.5\textwidth]{img/frontend/proposte:nuova.png}
    \caption{Form di creazione proposta. I campi extra per proposte nella categoria "Ambiente" includono campi testuali, numerici e un widget cartografico per il disegno dell'area di intervento.}
    \label{fig:proposte-nuova}
\end{figure}

\begin{figure}[H]
    \centering
    \includegraphics[width=0.9\textwidth]{img/frontend/proposte.png}
    \caption{Vista di dettaglio di una proposta cittadina. Evidenzia la mappa dell'area interessata, i dettagli tecnici e il pannello di votazione sulla destra.}
    \label{fig:proposte}
\end{figure}

\begin{figure}[H]
    \centering
    \includegraphics[width=0.8\textwidth]{img/frontend/sondaggi.png}
    \caption{Interfaccia di un sondaggio attivo. Mostra domande a risposta multipla e il feedback visivo immediato sulla distribuzione dei voti correnti.}
    \label{fig:sondaggi}
\end{figure}


\section{Area Personale e Profilo}

\begin{figure}[H]
    \centering
    \includegraphics[width=0.9\textwidth]{img/frontend/profilo:dati-personali.png}
    \caption{Sezione di Gestione Dati Personali e Sicurezza Account.}
    \label{fig:profilo-dati-personali}
\end{figure}

\begin{figure}[H]
    \centering
    \includegraphics[width=0.9\textwidth]{img/frontend/profilo:mie-proposte.png}
    \caption{Elenco delle proposte create dall'utente con i relativi stati di avanzamento.}
    \label{fig:profilo-mie-proposte}
\end{figure}

\begin{figure}[H]
    \centering
    \includegraphics[width=0.9\textwidth]{img/frontend/profilo:preferiti.png}
    \caption{Sezione Preferiti, raccoglie le proposte salvate dall'utente.}
    \label{fig:profilo-preferiti}
\end{figure}

