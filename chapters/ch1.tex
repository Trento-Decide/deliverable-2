% !TeX spellcheck = it_IT
\chapter{Web API}

Le API sono state documentate seguendo le specifiche openapi3, e la documentazione è consultabile pullicamente all'indirizzo: \url{https://app.swaggerhub.com/apis/tndecide/API/1.0}.

\section{Struttura e note sulle API}
Le rotte descritte nel seguente file .yaml sono sintetizzabili in:
\begin{itemize}
	\item \textbf{/auth} - login, register, provider SPID/CIE
	\item \textbf{/utente} - delete account
	\item \textbf{/proposte} - CRUD proposte, bozze, pubblicazione, voto, preferiti
	\item \textbf{/sondaggi} - lista, dettaglio, creazione (admin), voto
	\item \textbf{/cerca} - ricerca globale
	\item \textbf{/config} - categorie, form schema, stati
	\item \textbf{/dashboard} - statistiche (admin)
\end{itemize}

di seguito alcune note su implementazioni rilevanti.

\subsection{/auth}
\begin{itemize}
	\item \textbf{/login} - JWT stateless, al momento del login viene fornito all'utente un token d'accesso
	\item \textbf{/provider} - rotta placeholder per emulare SPID/CIE, fornisce un token fittizio
\end{itemize}

\subsection{/proposte}
Struttura molto ricca di endpoints, implementa dalla bozza al versioning del sistema delle proposte del software.
\begin{itemize}
	\item \textbf{CRUD proposals} - gestione di creazione / visualizzazione / modifica ed eliminazione delle proposte.
	\item \textbf{autorizzazioni} - Per ogni operazione relativa a modifica ed eliminazione viene richiesto dimostrare la paternità dell'oggetto.
	\item \textbf{versionamento} - per ogni PUT/:id il sistema registra una nuova modifica e resetta i voti a 0.
\end{itemize}

\subsection{/cerca}
Ricerca combinata su proposte e sondaggi tramite un unico endpoint. I risultati arrivano misti, discriminati successicamente dal campo \texttt{type}.

\subsection{/config}
Endpoint read-only. Utile al frontend nella popolazione dei form dinamici.
\begin{itemize}
	\item \textbf{form schema} - ogni categoria porta con sé uno schema JSON (\texttt{formSchema}) che descrive i campi aggiuntivi da mostrare nella creazione di una proposta. Il frontend li renderizza senza conoscerli a priori.
\end{itemize}

\subsection{/dashboard}
Singola rotta completamente riservata all'utente admin, restituisce le statistiche del sistema.
\begin{itemize}
	\item \textbf{protezione} - controllo doppio: JWT e verifica del ruolo \textbf{admin}.
\end{itemize}


\section{Contenuto esteso documentazione}
La specifica delle API è disponibile nel repository al link: \url{https://github.com/Trento-Decide/trento-decide}.

\lstinputlisting{oas3.yaml}

